\documentclass[letta4 paper]{article}
% Set target color model to RGB
\usepackage[inner=2.0cm,outer=2.0cm,top=2.5cm,bottom=2.5cm]{geometry}
\usepackage{setspace}
\usepackage[rgb]{xcolor}
\usepackage{verbatim}
\usepackage{subcaption}
\usepackage{amsgen,amsmath,amstext,amsbsy,amsopn,tikz,amssymb,tkz-linknodes}
\usepackage{fancyhdr}
\usepackage[colorlinks=true, urlcolor=blue,  linkcolor=blue, citecolor=blue]{hyperref}
\usepackage[colorinlistoftodos]{todonotes}
\usepackage{rotating}
\usepackage{listings}
\usepackage{algorithm}
\usepackage{algorithmic}
\lstset{
%	language=bash,
	basicstyle=\ttfamily
}

\newcommand{\ra}[1]{\renewcommand{\arraystretch}{#1}}

\newtheorem{thm}{Theorem}[section]
\newtheorem{prop}[thm]{Proposition}
\newtheorem{lem}[thm]{Lemma}
\newtheorem{cor}[thm]{Corollary}
\newtheorem{defn}[thm]{Definition}
\newtheorem{rem}[thm]{Remark}
\numberwithin{equation}{section}
\graphicspath{ {./img/} }

\newcommand{\homework}[6]{
   \pagestyle{myheadings}
   \thispagestyle{plain}
   \newpage
   \setcounter{page}{1}
   \noindent
   \begin{center}
   \framebox{
      \vbox{\vspace{2mm}
    \hbox to 6.28in { {\bf F1TENTH Autonomous Racing \hfill {\small (#2)}} }
       \vspace{6mm}
       \hbox to 6.28in { {\Large \hfill #1  \hfill} }
       \vspace{6mm}
       \hbox to 6.28in { {\it Instructor: {\rm #3} \hfill Name: {\rm #5}, StudentID: {\rm #6}} }
       %\hbox to 6.28in { {\it T\textbf{A:} #4  \hfill #6}}
      \vspace{2mm}}
   }
   \end{center}
   \markboth{#5 -- #1}{#5 -- #1}
   \vspace*{4mm}
}


\newcommand{\problem}[3]{~\\\fbox{\textbf{Problem #1: #2}}\hfill (#3 points)\newline}
\newcommand{\subproblem}[1]{~\newline\textbf{(#1)}}
\newcommand{\D}{\mathcal{D}}
\newcommand{\Hy}{\mathcal{H}}
\newcommand{\VS}{\textrm{VS}}
\newcommand{\solution}{~\newline\textbf{\textit{(Solution)}} }

\newcommand{\bbF}{\mathbb{F}}
\newcommand{\bbX}{\mathbb{X}}
\newcommand{\bI}{\mathbf{I}}
\newcommand{\bX}{\mathbf{X}}
\newcommand{\bY}{\mathbf{Y}}
\newcommand{\bepsilon}{\boldsymbol{\epsilon}}
\newcommand{\balpha}{\boldsymbol{\alpha}}
\newcommand{\bbeta}{\boldsymbol{\beta}}
\newcommand{\0}{\mathbf{0}}


\usepackage{booktabs}



\begin{document}

	\homework {Lab 1: Introduction to ROS}{Due Date:}{INSTRUCTOR}{}{STUDENT NAME}{ID}
	\thispagestyle{empty}
	% -------- DO NOT REMOVE THIS LICENSE PARAGRAPH	----------------%
	\begin{table}[h]
		\begin{tabular}{l p{14cm}}
		\raisebox{-2cm}{\includegraphics[scale=0.5, height=2.5cm]{f1_stickers_02} } & \textit{This lab and all related course material on \href{http://f1tenth.org/}{F1TENTH Autonomous Racing} has been developed by the Safe Autonomous Systems Lab at the University of Pennsylvania (Dr. Rahul Mangharam). It is licensed under a \href{https://creativecommons.org/licenses/by-nc-sa/4.0/}{Creative Commons Attribution-NonCommercial-ShareAlike 4.0 International License.} You may download, use, and modify the material, but must give attribution appropriately. Best practices can be found \href{https://wiki.creativecommons.org/wiki/best_practices_for_attribution}{here}.}
		\end{tabular}
	\end{table}
	% -------- DO NOT REMOVE THIS LICENSE PARAGRAPH	----------------%
	
	\noindent \large{\textbf{Course Policy:}} Read all the instructions below carefully before you start working on the assignment, and before you make a submission. All sources of material must be cited. The University Academic Code of Conduct will be strictly enforced.

	\section{Workspaces and Packages}

	
	\subsection{Written Questions}
	\begin{enumerate}
		\item A CMakeList is the main CMake file to build a ROS package. It contains a set of instruction describing the source files and target files of a project.
CMakeList is used to create a make file, where make file is a build system which builds the code to an executable. So, CMake List is basically a generator of build systems. 

		\item Yes, CMakeList is also required for Python in ROS. No executable object will be created. 
		\item You will run it in your workspace root.
		\item The purpose of setupbash is to configure Environment Variables to the  correct path. The 1st setup.bash files sets general ROS Environment Variable, while the second setup.bash file sets path to that specific workspace. 
		
	\end{enumerate}{}

	\section{Publishers and Subscribers}
	
	\subsection{Written Questions}
	\begin{enumerate}
		\item A nodehandle object basically represents the node itself. It controls the initialization and shutdown of a node. Also, you can specify a namespace relative to the nodehandle.
It is possible to have multiple nodehandle objects in a node.

		\item ros::SpinOnce checks for callbacks and service calls once when you call it. 
Ros::spin, on the other hand, repeatedly calls ros::SpinOnce so it spin until the node dies. 

		\item Answer here
		\item ros::rate allows you to call a function periodically. It takes account of the varying execution time and adjust its sleep time accordingly. 
		\item Answer here
	\end{enumerate}{}
			
	\section{ Implementing Custom Messages}
	
	\subsection{Written Questions}
	\begin{enumerate}
		\item This is because C++ doesn’t understand .msg files. So Cmake created a .h header file to represent the msg file in what C++ understands. 
		\item Header is not a primitive type, and is instead resolved to std_msgs/ Header. It is possible to include in my application since I have access to std_msgs as well. It provide data on a sequential id, time stamp and frame id. 
	\end{enumerate}{}

	\section{Recording and Publishing Bag Files}
	\subsection{Written Questions}
	
	\begin{enumerate}
		\item The bag file is saved in the directory where rosbag record is called. You can change where it’s saved by calling rosbag record from the directory you desire. 
		\item The bag file is saved in the directory specified by the argument ‘’agrs”.
Yes. This can be done by filling node arguments in the launch file. The argument “args” represent the path the bag file would be saved. 
 
	\end{enumerate}{}




\end{document} 
